\newpage
\section{Anforderungen}
\label{sec:Anforderungen}
Wie eingangs bereits beschrieben sollen lediglich die beiden Anwendungen für den Administrator bzw. den Partygast, im Folgenden "`Admin"' bzw. "`VoteApp"' genannt, realisiert werden. In den nachfolgenden Abschnitten werden die Anforderungen an die beiden Anwendungen sowie Entwürfe der Benutzeroberflächen vorgestellt. Eine gemeinsame Anforderung ist dabei die Lokalisierung in deutscher und englischer Sprache.

\subsection{Admin}
Das Anwendungsfalldiagramm (Abb. \ref{fig:AdminUseCase}) stellt die Anforderungen an die Admin-Anwendung dar:

\begin{figure}[H]
\centering
\includegraphics[width=0.7\linewidth]{Bilder/AdminUseCase}
\caption{Anwendungsfalldiagramm für den Admin\protect\footnotemark}
\label{fig:AdminUseCase}
\end{figure}
\footnotetext{Das Anwendungsfalldiagramm haben wir aus unserer MWA-Dokumentation übernommen.}

Der Admin muss die Party starten und stoppen sowie Informationen über die Party einsehen können. Darüber hinaus soll er in der Lage sein, die Songsammlung zu verwalten.

Wir haben uns dazu entschieden, je eine eigene Oberfläche für den Bereich der Party- und der Songverwaltung zu entwickeln. Daneben gibt es einen weiteren Bereich zur Auswahl der Sprache. Diese Funktionalitäten werden in einem gemeinsamen Tab-Layout angeboten.
  
\begin{figure}[H]
\centering
\includegraphics[width=0.85\linewidth]{Bilder/MockParty}
\caption{Entwurf der Partyverwaltung}
\label{fig:MockParty}
\end{figure}

Die Oberfläche zur Verwaltung der Party (Abb. \ref{fig:MockParty}) verfügt über Buttons zum Starten und Stoppen der Party sowie über eine Auflistung der Songs der aktuellen Playlist inkl. deren Votings. Des Weiteren werden Informationen über eine laufende Party wie Startzeit, Anzahl der Partygäste und der durchschnittliche Drunken-Index dargestellt.

\begin{figure}[H]
\centering
\includegraphics[width=1\linewidth]{Bilder/MockSongVerwaltung}
\caption{Entwurf der Songverwaltung}
\label{fig:MockSongVerwaltung}
\end{figure}

Die Oberfläche zur Verwaltung der Songs (Abb. \ref{fig:MockSongVerwaltung}) ist zweigeteilt. Die linke Seite beinhaltet eine Auflistung aller Songs der Songsammlung und ein Eingabefeld zur Suche. Darüber hinaus kann ein ausgewählter Song mittels entsprechender Buttons gelöscht bzw. ein neuer erstellt werden. Die rechte Seite 
beinhaltet detaillierte Informationen zu einem ausgewählten Song. Diese lassen sich in den Eingabefeldern bearbeiten und speichern.

\begin{figure}[H]
\centering
\includegraphics[width=0.6\linewidth]{Bilder/MockSprachen}
\caption{Entwurf der Sprachumschaltung}
\label{fig:MockSprachen}
\end{figure}

Der dritte Reiter des Tab-Layouts umfasst zwei Buttons zur Wahl der Sprache (Abb. \ref{fig:MockSprachen}).

\subsection{VoteApp}
Das Anwendungsfalldiagramm (Abb. \ref{fig:PartyPeopleUseCase}) stellt die Anforderungen an die Anwendung für den Partygast dar:

\begin{figure}[H]
\centering
\includegraphics[width=0.8\linewidth]{Bilder/PartyPeopleUseCase}
\caption{Anwendungsfalldiagramm für den Partygast\protect\footnotemark}
\label{fig:PartyPeopleUseCase}
\end{figure}
\footnotetext{Das Anwendungsfalldiagramm haben wir aus unserer MWA-Dokumentation übernommen.}

Dem Partygast sollen die Playlist und der aktuelle Song angezeigt werden. Außerdem soll er einzelne Songs voten und seinen Drunken-Index (DI) messen können.

\begin{figure}[H]
\centering
\includegraphics[width=0.45\linewidth]{Bilder/MockPartyPeopleClient}
\caption{Entwurf der Oberfläche der Anwendung für den Partygast}
\label{fig:MockPartyPeopleClient}
\end{figure}

Die Anwendung besteht aus einer Hauptoberfläche (Abb. \ref{fig:MockPartyPeopleClient}). Diese beinhaltet alle Songs der Playlist mit je zwei Buttons für das Voting sowie ein gesondertes Feld für den aktuell spielenden Song. Darüber hinaus gibt es einen Button, der zur Eingabe des Drunken-Index führt.

\begin{figure}[H]
\centering
\includegraphics[width=0.45\linewidth]{Bilder/MockDiSlider}
\caption{Dialog zur Eingabe des Drunken-Index}
\label{fig:MockDiSlider}
\end{figure}

Bei Betätigung des entsprechenden Buttons öffnet sich ein Dialog (Abb. \ref{fig:MockDiSlider}). Dieser besteht aus einem Schieberegler zur Einstellung des Drunken-Index auf einer Skala von \textit{nüchtern} über \textit{lustig} bis \textit{betrunken}. Mittels zweier Buttons lässt sich der eingestellte Wert versenden oder die Aktion abbrechen.