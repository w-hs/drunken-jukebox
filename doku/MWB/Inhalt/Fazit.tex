\section{Schlussteil}
\subsection{Probleme}

\subsection{Ausblick}
Im Verlauf des Projekts haben sich weitere Ideen entwickelt, die wir leider aus zeitlichen Gründen nicht mehr umsetzen konnten. Anwendungsübergreifend hätten wir gerne Tests implementiert, um diesbezüglich Erfahrung im GWT-Umfeld zu sammeln. Für die Admin-Applikation wäre die Möglichkeit eines batchartigen Musikimports mehr als wünschenswert. Auf Seiten der VoteApp gibt es eine ganze Reihe von Verbesserungs- und Erweiterungsideen. Grundsätzlich sollte die VoteApp für die Nutzung auf einem Smartphone optimiert werden. Im diesem Zusammenhang könnten weitere visuelle Effekte beispielsweise für die Veränderung der Song-Rangliste hinzugenommen werden, um die Anwendung ansprechender zu gestalten. Darüber hinaus muss die Anwendung auf den wichtigsten mobilen Betriebssysteme lauffähig sein, was beispielsweise den Einsatz von PhoneGap auf den Plan ruft. Die Bedienbarkeit muss dazu so angepasst werden, dass sie allen Design-Richtlinien entspricht. Beispielweise könnte in diesem Zusammenhang der Dialog, der zur Eingabe des DI-Wertes dient, durch dynamische Elemente ersetzt werden, sodass keine zusätzlichen Fenster notwendig sind. Bezüglich des Drunken-Index besteht des Weiteren die Idee statt des Sliders den DI auf anderem Wege zu ermitteln. Eine Art Spiel, anhand dessen sich ein Wert auf der Betrunkenheitsskala ermitteln lässt, wäre vorstellbar. Unabhängig davon sollte jedoch in jedem Fall der Partygast motiviert werden seinen DI regelmäßig einzureichen. Es wäre ein Punktesystem denkbar, bei dem es für die Eingabe des DIs Punkte gibt, die bei Abgabe eines Votes verbraucht werden. Neben den vorgestellten Ideen sind selbstredend noch viele weitere Innovationen vorstellbar.

\subsection{Fazit}
Im Verlauf des MWB-Projekts wurden zwei Anwendungen in GWT implementiert, die über einen Proxy mit dem Backend auf Basis von Deployd als Alternative zum WildFly kommunizieren. Sowohl die Anwendung für den Administrator als auch die App für die Partygäste erfüllen alle eingangs spezifizierten Anforderungen. Unter der Verwendung von MVP haben wir zahlreiche eigene Widgets entwickelt, die in beiden Anwendungen genutzt werden. Die Vorteile von MVP sind allerdings aufgrund der geringen Komplexität unseres Projekts nicht deutlich geworden. 

Die Verwendung von GWT und den dazu gehörigen Erweiterungen verlief für ein so großes Framework erstaunlicherweise problemlos.
So brachte beispielsweise ein Upgrade der GWT-Version von 2.6 auf 2.7 zur Mitte unseres Projekts keinerlei Probleme mit sich. Zudem stehen eine sehr große Auswahl der wichtigsten Oberflächenkomponenten bereits im Standard zur Verfügung. Darüber hinaus verlief der Einsatz von JSNI  ohne erwähnenswerte Komplikationen. Ebenso gut hat uns das Grundprinzip von GSS gefallen, wobei sich teilweise der Eindruck verstärkt hat, dass es nicht zu Ende gedacht ist. Zuweisungen und das Durchreichen von Styles ist gewöhnungsbedürftig und wir hätten uns gewünscht, dass es möglich ist, mehrere CSS-Definitionen auf einer höheren Abstraktionsebene zu überspannen.

Insgesamt hat das Projekt gezeigt, dass GWT ein gutes Tool ist, welches zum Einsatz in produktiven Umgebungen geeignet ist. Wir sind abschließend mit unserem Projekt und den Möglichkeiten von GWT zufrieden.

